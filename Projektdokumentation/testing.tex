\chapter{Indledning}

\chapter{Kravspecifikation}
\section{Godkendelsesformular}
\begin{table}[h!]
\label{tab:tabel2}
\begin{tabular}{| l | >{\raggedright\arraybackslash}p{12cm} |}
   \hline
   \textbf{Forfattere} & Line, Mette, Brian, Mohamed, Khaled og Ida\\ \hline
   \textbf{Godkendes af:} & Samuel Alberg Thrysøe\\ \hline
   \textbf{Antal sider:} & \\ \hline
   \textbf{Kunde:} & IHA\\ \hline
\end{tabular}
\end{table}
\textbf{Ved underskrivelse af dette dokument accepteres det af begge parter, som værende kravene til udviklingen af det ønskede system.}
\newline
\textbf{Sted og dato:}\\
\\
\\
\begin{table}
[h!]
\begin{tabular}{ l lllllllll l}
--------------------------------------&&&&&&&&&&--------------------------------------\\ 
Kundens underskrift &&&&&&&&&&Leverandørens underskrift\\
\end{tabular}
\end{table}
\section{Indledning}
Denne kravspecifikation er blevet udarbejdet på baggrund af krav fra kunden, samt hvad leverandøren finder muligt. Kravspecifikationens formål er at specificere de krav der er til produktet.
\newpage

\section{Systembeskrivelse}
Blodtryksmålersystemet ønskes at kunne måle blodtryk, EKG og iltmætning for en patient. Ud fra blodtrykket findes systolisk, diastolisk og middeltryks værdier, dette gøres ved at finde den maksimale værdi (systole), den minimale værdi (diastole) og det bestemte intergrale (middeltrykket) af blodtrykskurven. Ud fra EKG-signalet kan pulsen bestemmes, dette gøres ved at bestemme antallet af R-takker på et minut. 
\newpage

\section{Aktør-kontekst diagram}
\begin{figure}[h!]
\includegraphics[width =0.65\textwidth , center]{billeder/Aktorkontekst.jpg}
\caption{\textbf{Aktør-kontekst diagram}}
\end{figure}

Af dette diagram ses de interagerende aktører: \textit{Sundhedsfaglig personale}, \textit{Transducer}, \textit{EPJ database}, \textit{Personale database} og \textit{Servicemedarbejder}.\\ Herunder er der en detaljeret beskrivelse af hver aktør.

\begin{table}[h!]
\begin{tabular}{| >{\raggedright\arraybackslash}p{3cm} | >{\raggedright\arraybackslash}p{12cm} |}
   \hline
   Navn: & Sundhedsfaglig personale\\ \hline
   Type: & Primær aktør \\ \hline
   Beskrivelse: & Det sundhedsfaglige personale er aktøren, der sætter måleudstyret til transduceren, samt starter målingen. Det er det sundhedsfaglige personale, der interagerer med systemet og dermed har tilgang til de viste målinger på brugergrænsefladerne (startskærm og hovedskærm).\\ \hline
\end{tabular}
\end{table}


\begin{table}[h!]
\begin{tabular}{| >{\raggedright\arraybackslash}p{3cm} | >{\raggedright\arraybackslash}p{12cm} |}
   \hline
   Navn: & Transducer\\ \hline
   Type: & Sekundær aktør \\ \hline
   Beskrivelse: & Transduceren er kilden til måleresultaterne, og dermed fungerer som patienten. Måleresultater opnås ved, at disse data sendes ind i systemet igennem hardwaren.\\ \hline
\end{tabular}
\end{table}


\begin{table}[h!]
\begin{tabular}{| >{\raggedright\arraybackslash}p{3cm} | >{\raggedright\arraybackslash}p{12cm} |}
   \hline
   Navn: & Personale database\\ \hline
   Type: & Sekundær aktør \\ \hline
   Beskrivelse: & Personale database er der, hvori det sundhedsfaglige personales login informationer obevares, hvilket benyttes til at tilgå systemet. \\ \hline
\end{tabular}
\end{table}


\begin{table}[h!]
\begin{tabular}{| >{\raggedright\arraybackslash}p{3cm} | >{\raggedright\arraybackslash}p{12cm} |}
   \hline
   Navn: & EPJ database\\ \hline
   Type: & Sekundær aktør \\ \hline
   Beskrivelse: & EPJ database er den database, hvor patientdata ligger, samt der hvori analyseresultaterne, der opnås ved målingerne i systemet, samt signalerne bliver gemt. Disse data er grafer for EKG, arterietryk, iltmætnings, samt talværdier for puls, systole, diastole, middeltryk og iltmætningen.\\ \hline
\end{tabular}
\end{table}

\begin{table}[h!]
\begin{tabular}{| >{\raggedright\arraybackslash}p{3cm} | >{\raggedright\arraybackslash}p{12cm} |}
   \hline
   Navn: & Servicemedarbejder\\ \hline
   Type: & Primær aktør \\ \hline
   Beskrivelse: & Servicemedarbejderen er aktøreren der igangsætter og foretager kalibreringen.\\ \hline
\end{tabular}
\end{table}

\newpage

\section{Use cases}
\begin{figure}[h!]
\includegraphics[width =0.7\textwidth , center]{billeder/UCdiagram}
\caption{\textbf{Use case diagram}}
\end{figure}

De fire Use cases er: \textit{Kalibrér apparat}, \textit{Foretag måling}, \textit{Alarmér} og \textit{Stop måling}. Hver enkel af disse Use cases beskrives detaljeret herunder i et fully-dressed Use case skema\\
Systemet består af en computer med programkode, en NI-DAQ og en Analog Discovery. Derudover har vi hardware, hvilket består af et lavpasfilter, en forstærker og en transducer.
\\
Systemet gør det muligt at hente data fra transduceren. Her går data fra transduceren igennem forstærkeren, hvilken der bliver forsynet af spænding fra Analog Discovery, som er forsynet af spænding fra computeren. Fra forstærkeren gør signalet videre til lavpasfilteret og derefter ind i NI-DAQ, som så sender dataen videre til computeren. 
\\ 

"Signalets vej" og opbygning beskrives\\\\


\begin{table}[H]
\caption{Use case 1}\label{tab:tabel3}
\begin{tabular}{| l | >{\raggedright\arraybackslash}p{11cm} |}
   \hline
   \textbf{Use case 1} & \textbf{Kalibrér apparat}\\ \hline
   Mål: & Få kalibreret apparatet \\ \hline
   Initiering: & Startes af Servicemedarbejder\\ \hline
   Aktører:& Servicemedarbejder (primær)\\ \hline
   Referencer: & - \\ \hline
   Samtidige forekomster: & én kalibrering pr. apparat \\\hline
   Forudsætninger: & Blodtryksmålersystemet er tændt og tilsluttet kalibreringsudstyret.\\ \hline
   Resultat:& Apparatet er kalibreret\\ \hline
   Hovedscenarie:& 
1. Servicemedarbejder starter kalibreringen. \newline
2. Systemet foretager kalibrering. \\\hline
Udvidelse/undtagelser: & - \\\hline
\end{tabular}
\end{table}


\begin{table}[H]
\caption{Use case 2}\label{tab:tabel3}
\begin{tabular}{| l | >{\raggedright\arraybackslash}p{11cm} |}
   \hline
   \textbf{Use case 2} & \textbf{Foretag måling}\\ \hline
   Mål: & Den valgte patients målinger foretages\\ \hline
   Initiering: & Startes af Sundhedsfaglig personale\\ \hline
   Aktører:& Sundhedsfaglig personale (primær), Personale database (sekundær), EPJ database(sekundær), Transducer (sekundær)\\ \hline
   Referencer: & Use case 2 \\ \hline
   Samtidige forekomster: & Én sundhedsfaglig person og én transducer pr. system \\\hline
   Forudsætninger: & VPN, Personale database og EPJ databasen er tilsluttet korrekt\\ \hline
   Resultat:& Målingerne for den valgte patient er foretaget\\ \hline
   Hovedscenarie:& 
1. Sundhedsfaglig personale logger på ved at indtaste brugernavn og kode. \newline
   \textit{$[$Undtagelse 1: Brugernavn og/eller kode indtastet forkert$]$}\newline
2. Besked: "Logget på" vises  \newline
3. Liste med patienter kommer frem\newline
4. Den ønskede patient vælges \newline
5. Sundhedsfaglig personale starter målingen \newline
6. Systemet indhenter data fra transduceren og måler tiden for hvor lang tiden målingen foretages\newline
7. EKG og arterietryk præsenteres kontinuert på hver sin graf. Puls, systole, diastole og middeltryk vises som talværdier. Data gemmes automatisk kontinuert i EPJ database. \newline
\textit{$[$Udvidelse 1: Slå digitalt filter til/fra$]$}\newline
\textit{$[$Udvidelse 2: Juster systolens/diastolens grænseværdi$]$}\newline
8. Sundhedsfaglig personale trykker på "Nulpunks justering"\newline
9. Systemet starter nulpunkts justeringen og giver besked: "Nulpunkts justeringen er fuldent" \\\hline
Udvidelse/undtagelser: & $[$Undtagelse 1: Brugernavn og/eller kode indtastet forkert$]$\newline
1.1 Besked: "Brugernavn og/eller kode indtastet forkert"\newline
1.2 Use case 3 starter forfra \newline\newline
$[$Udvidelse 1: Slå digitalt filter til/fra$]$\newline 
1.1 Sundhedsfaglig personale vælger "Digitalt filter OFF" \newline
1.2 Systemet slår det digitale filter fra\newline
1.3 Sundhedsfaglig personale vælger "Digitalt filter ON"\newline
1.4 Systemet slår det digitale filter til\newline\newline
$[$Udvidelse 2: Juster systolens/diastolens grænseværdi$]$\newline
2.1 Sundhedsfaglig personale justerer grænseværdierne for systole og/eller diastole.
\\\hline
\end{tabular}
\end{table}


\begin{table}[H]
\caption{Use case 4}\label{tab:tabel3}
\begin{tabular}{| l | >{\raggedright\arraybackslash}p{11cm} |}
   \hline
   \textbf{Use case 3} & \textbf{Alarmér}\\ \hline
   Mål: & Få startet alarmeringen ved overskridelse af grænseværdi \\ \hline
   Initiering: & Systemet starter denne Use case\\ \hline
   Aktører:& Sundhedsfaglig personale (sekundær)\\ \hline
   Referencer: & Use case 3 \\ \hline
   Samtidige forekomster: & - \\\hline
   Forudsætninger: & Målingen i Use case 2: Foretag måling, er kørt succesfuldt \\ \hline
   Resultat:& Alarmen starter\\ \hline
   Hovedscenarie:& 
1. Grænseværdi overskrides \newline
2. Alarm starter med lyd og tallet, hvis grænseværdi er overskredet, blinker.\newline
    \textit{$[$Udvidelse 1: Anden grænseværdi overskrides$]$} \newline
    \textit{$[$Udvidelse 2: Udsæt alarm$]$ }
\\\hline
Udvidelse/undtagelser: & $[$Udvidelse 1: Anden grænseværdi overskrides$]$ \newline
1.1. Endnu en grænseværdi overskrides\newline
1.2. Lyden fra første alarm fortsætter. Det nye tal som har overskredet grænseværdien blinker ligeledes.\newline
1.3 Use case afsluttet.\newline\newline
$[$Udvidelse 2: Udsæt alarm$]$\newline
2.1 Sundhedsfaglig person udsætter alarm\newline
2.2 Systemet stopper alarmens lyd i et minut
\\\hline
\end{tabular}
\end{table}


\begin{table}[H]
\caption{Use case 4}\label{tab:tabel3}
\begin{tabular}{| l | >{\raggedright\arraybackslash}p{11cm} |}
   \hline
   \textbf{Use case 4} & \textbf{Stop måling}\\ \hline
   Mål: &  Få stoppet målingen og logget\\ \hline
   Initiering: & Startes af Sundhedsfaglig personale \\ \hline
   Aktører: & Sundhedsfaglig personale (primær) \\ \hline
   Referencer: & Use case 2\\ \hline
   Samtidige forekomster: & - \\\hline
   Forudsætninger: & Use case 2: Foretag måling, er kørt succesfuldt\\ \hline
   Resultat:& Signalet er stoppet, sundhedsfaglig personale er logget ud og vendt tilbage til startskærm.\\ \hline
   Hovedscenarie:& 
1. Sundhedsfaglig personale stopper måling\newline
2. Systemet stopper målingen.\newline 
3. Sundhedsfaglig personale logger ud \\\hline
Udvidelse/undtagelser: & -\\\hline
\end{tabular}
\end{table}



\newpage 
\newpage 
\newpage
\newpage



\section{Ikke-funktionelle krav}
De ikke-funktionelle krav er opsat efter FURPS+ metoden. De er prioriteret efter MoSCoW metoden:
\begin{itemize}
\item \textbf{M}ust (skal være med)
\item \textbf{S}hould (bør være med, hvis muligt)
\item \textbf{C}ould (kunne have med, hvis det ikke går i vejen for noget andet)
\item \textbf{W}on't/\textbf{W}ould (tager det ikke med nu, men kan komme med i fremtidige opdateringer)
\end{itemize}

\subsection{FURPS+ med MoSCoW}
\begin{enumerate}
\item \textbf{Functionality}
\begin{enumerate}
\item (\textbf{M}) Programmet skal have et digitalt filter til udglatning af blodtrykssignal
\item (\textbf{M})Programmet skal give alarm når grænseværdier overskrides med lyd og hvor den overskredede grænseværdi blinker på skærmen.
\item (\textbf{M}) Programmet skal kunne gemme blodtrykssignalet i en database.
\end{enumerate}
\item \textbf{Usability}
\begin{enumerate}
\item (\textbf{S}) Programmet skal have to window forms: startskærm, der fungerer som  EPJ systemet og hovedskærm, hvilken fungerer som selve blodtryksmålerens grænseflade.
\item (\textbf{M}) Programmet skal have en "Login" knap på startskærmen
\item (\textbf{M}) Programmet skal have en "Kalibrering" knap på startskærmen
\item (\textbf{M}) Sundhedsfagligt personale skal kunne ændre "devicename/enhedsnavn" i dropdown på startskærmen
\item (\textbf{S}) Programmet skal indeholde en dropdown, hvor patienten kan vælges, på startskærmen
\item (\textbf{M}) Programmet skal have en "Nulpunkts indstilling" knap på hovedskærmen
\item (\textbf{M}) Programmet skal have en knap til at slå det digitale filter fra og til på hovedskærmen
\item (\textbf{M}) Programmet skal have knapper til at justere systolisk og diastolisk grænseværdi-intervaller op og ned, på hovedskærmen
\item (\textbf{M}) Programmet skal have en "Udsæt alarm" knap på hovedskærmen
\item (\textbf{M}) Programmet skal have en "Tænd" knap på hovedskærmen.
\item (\textbf{M}) Programmet skal have en "Sluk" knap på hovedskærmen
\item (\textbf{M}) Programmet skal have en "Afbryd" knap på hovedskærmen.
\item (\textbf{M}) Teksten på startskærmen skal kunne læses fra 2 meters afstand ved synsstyrke i intervallet på +/-1
\item (\textbf{M}) Teksten og graferne på hovedskærmen skal kunne læses fra 2 meters afstand ved synsstyrke i intervallet på +/-1 
\item (\textbf{M}) Programmet skal præsentere data på grafer på følgende måde (Se afsnit nedenfor)
\begin{itemize}
\item EKG vises i lysegrøn
\item Arterietryk vises i rød
\item Iltmætning/saturation i lyseblå
\end{itemize}
\item (\textbf{M}) Programmet skal præsentere data i tal på følgende måde (Se afsnit nedenfor)
\begin{itemize}
\item Hjertefrekvens i lysegrøn
\ Systolisk samt diastolisk tryk i rødt, ligeledes middelblodtrykket i parentes under i rødt.
\end{itemize}
\begin{figure}[h!]
\includegraphics[width =0.4\textwidth , center]{billeder/skitseStart}
\caption{Skitse af startskærmen, hvilken repræsenterer EPJ systemet}
\end{figure}
\begin{figure}[h!]
\includegraphics[width =1.0\textwidth , center]{billeder/skitseHoved}
\caption{Skitse af hovedskærmen, hvilken repræsenterer en blodtryksmålers brugerflade}
\end{figure}
\end{enumerate}
\item \textbf{Reliability}
\begin{enumerate}
\item (\textbf{S}) INGEN RELIABILITY KRAV ENDNU
\end{enumerate}
\item \textbf{Performance}
\begin{enumerate}
\item (\textbf{S}) Tiden der går før måling af data påbegynder / vises i grafer må maksimalt være 2 sek.
\item (\textbf{M}) Tiden der går fra at data, herunder puls, diastolisk tryk, systolisk tryk, middeltryk og iltmætning, er analyseret til at data'en er gemt i EPJ database må være 2 sek. med en tolerance på +/-15\% 
\item (\textbf{S}) Ved justering af grænseværdi for systole og diastole ændres grænseværdien 2.5mmHg op eller ned.                                                                                                                                                                                                                                                                                                                                                                                                                                                                                                                                                                                                                                                                                               
\end{enumerate}
\item \textbf{Supportability}
\begin{enumerate}
\item (\textbf{M}) Softwaren skal være opbygget efter trelagsmodellen (Data-View-Model)
\end{enumerate}
\item \textbf{+ Test conditions}
\begin{enumerate}
\item (\textbf{M}) Der skal være adgang til en computer med Windows 7, 8 eller 10 - computeren skal minimum have 4 GB RAM.
\item (\textbf{M}) Der skal være adgang til en computer hvor National Instruments er installeret.
\end{enumerate}
\end{enumerate}
\chapter{Arkitektur og design}
Følgende beskriver arkitekturen for systemet herunder både hardware og software. 
Systemarkitektur er udviklingsrammen for den videre udvikling af design og implementering. Der vil igennem dette afsnit startes med at se systemet overordnet og hvorefter der arbejdes ned gennem systemet i mindre brudstykker. Der benyttes diagrammer for at kunne specificere og klarlægge systemkravene. Disse diagrammer beskrives desuden i tekst.
\section{Hardware design}
%\begin{figure}[h!]
%\includegraphics[width =1.0\textwidth , center]{billeder/BDD}
%\caption{\textbf{Block definition diagram. Dette diagram viser hardware delene i systemet, samt sammenhængen mellem disse.}}
%\end{figure}
%\begin{figure}[h!]
%\includegraphics[width =1.0\textwidth , center]{billeder/IBD}
%\caption{\textbf{Internal block diagram. Dette diagram viser signalerne imellem blokkene.}}
%\end{figure}

\subsection{Implementering}
\subsection{Modultest}
\section{Software design}
I dette afsnit beskrives softwaredesign på baggrund af systembeskrivelse og kravspecifikationen. DEnne beskrivelse opnås ved at der benyttes relevante diagrammer og modeller, hvilke kan beskrive softwaren. Overvejelser og valg, der er blevet gjort, i forbindelse med design og implementering af softwaren, vil i dette afsnit blive præsenteret.
\subsection{Problemidentifikation}
Først skal der klarlægges hvilke klasser som systemet skal bestå af, hvilket er det første skridt i processen. For at kunne identificere disse klasse udarbejdes en domænemodel, hvilken har sit udgangspunkt i Use cases. Det er i de konceptuelle klasser fra Use cases som indeholder den information som systemet skal holde styr på. Derfor findes de konceptuelle klasser i Use cases og disse indføres i domænemodellen som klasser. Domænemodellen opstilles derfor for at finde frem til hvad problemet er i softwaren i forhold til, hvad der skal holdes styr på.
\begin{figure}[h!]
\includegraphics[width =1.0\textwidth , center]{billeder/DM}
\caption{\textbf{Domænemodel for blodtryksmålersystemet.}}
\end{figure}
Denne domænemodel viser det sundhedsfaglige personales interaktion med systemet, samt hvilke handlinger der startes af denne interaktion. Det sundhedsfaglige personale udfører en handling, der medfører, at en række processer igangsættes i systemet. Disse processer sørger at hente data fra transduceren og EKG patient, for at starte beregningen af puls, systolisk, diastolisk og middeltryks værdierne, samt sørger for at disse data bliver gemt i en database.
\\
\subsubsection{Klasseidentifikation}
Ud fra domænemodellen kan en applikationsmodel opstilles. Denne model tager afsæt i domænemodellens
klasser. Dette betyder derfor at denne model således også tager udgangspunkt i alle Use cases.
\begin{figure}[h!]
\includegraphics[width =1.0\textwidth , center]{billeder/appModel}
\caption{\textbf{Applikationsmodel for blodtryksmålersystemet.}}
\end{figure}
\subsubsection{Metodeidentifikation}
\begin{figure}[h!]
\includegraphics[width =1.0\textwidth , center]{billeder/sdUC1}
\caption{\textbf{Sekvensdiagram for blodtryksmålersystemet. Denne viser adfærden for Use case 1 }}
\end{figure}

\begin{figure}[h!]
\includegraphics[width =1.0\textwidth , center]{billeder/sdUC2}
\caption{\textbf{Sekvensdiagram for blodtryksmålersystemet. Denne viser adfærden for Use case 2}}
\end{figure}

\begin{figure}[h!]
\includegraphics[width =1.0\textwidth , center]{billeder/sdUC3}
\caption{\textbf{Sekvensdiagram for blodtryksmålersystemet. Denne viser adfærden for Use case 3}}
\end{figure}

\begin{figure}[h!]
\includegraphics[width =1.0\textwidth , center]{billeder/sdUC4}
\caption{\textbf{Sekvensdiagram for blodtryksmålersystemet. Denne viser adfærden for Use case 4}}
\end{figure}

\subsection{Implementering}
\subsection{Unittest}
\section{Integrationstest}


\chapter{Accepttest}
\section{Indledning}
Accepttestene skal vise om kravene der er opstillet for blodtryksmålersystmet lever op til de standarder der er sat op for at produktet aktivt kan indgå i en hverdag på afdelingen.\\
Accepttestene er er opfølgning af kravsspecifikationen, hvilket sikre at alle krav er overholdt og dermed opnået.\\\\
Når der i feltet Godkendt er skrevet initialer for den der har udført testen, samt datoen for testens udførelse, betyder det at testen er godkendt.  \\

\section{Accepttest for funktionelle krav}
\subsection{Opstilling}
Billede indsættes - haves ikke endnu

\begin{table}[H]
\caption{Accepttest for Use case 1}\label{tab:tabel8}
\begin{tabular}{|>{\raggedright\arraybackslash}p{2.5cm}| >{\raggedright\arraybackslash}p{2.9cm} | >{\raggedright\arraybackslash}p{2.9cm} | >{\raggedright\arraybackslash}p{2.9cm} | >{\raggedright\arraybackslash}p{2.8cm} |}
   \hline
   \textbf{Use case 1: Kalibrer apparat} &\textbf{Test}& \textbf{Prækondition} & \textbf{Forventet resultat} & \textbf{Godkendt/ kommentar}\\ \hline
   Normalforløb:& Tryk på "Kalibrering" & Blodtryksmåleren er tændt og tilsluttet kalibreringsudstyret. & Systemet er kalibreret og besked: "Kalibreringen er fuldendt" vises på startskærmen & IKKE TESTBAR\\\hline
\end{tabular}
\end{table}

\begin{table}[H]
\caption{Accepttest for Use case 2}\label{tab:tabel8}
\begin{tabular}{|>{\raggedright\arraybackslash}p{2.5cm}| >{\raggedright\arraybackslash}p{2.9cm} | >{\raggedright\arraybackslash}p{2.9cm} | >{\raggedright\arraybackslash}p{2.9cm} | >{\raggedright\arraybackslash}p{2.8cm} |}
   \hline
   \textbf{Use case 2: Foretag måling} &\textbf{Test}& \textbf{Prækondition} & \textbf{Forventet resultat} & \textbf{Godkendt/ kommentar}\\ \hline
   Normalforløb:& Indtast brugernavn "anpe" og kode "1234" & Port valgt. VPN, Personale database og EPJ database er tilsluttet korrekt & Korrekt indtastning fuldendt & \\\hline
   &Tryk "Login" & Port valgt. VPN, Personale database og EPJ database er tilsluttet korrekt & Besked: "Logget på" og den sundhedsfaglige er dermed logget på &\\\hline
   &Tryk på patient dropdown på startskærm & En sundhedsfaglig er logget på & Liste med patienter kommer frem  & \\\hline
   & Vælg patienten "Arne Jensen" & Den sundhedsfaglige er logget på & Nyt vindue kommer frem: Hovedskærmen &\\\hline
   & Tryk på "Tænd"& Patient valgt og transduceren er tilsluttet & Systemet indhenter data fra transduceren og starter timer på hovedskærm. EKG, arterietryk og iltmætningskurve præsenteres kontinuert på hver sin graf. Puls, systole, diastole, middeltryk og iltmætning vises som talværdier på hovedskærmen. Data gemmes automatisk kontinuert i EPJ database & \\\hline
   & Tryk på "Nulpunkts justering" & Signalet er startet og kører & Systemet starter nulpunkts justeringen. Besked " Nulpunkts justering er fuldendt" vises på hovedskærmen &\\\hline
\end{tabular}
\end{table}



\begin{table}[H]
\caption{Accepttest for Use case 2}\label{tab:tabel8}
\begin{tabular}{|>{\raggedright\arraybackslash}p{2.5cm}| >{\raggedright\arraybackslash}p{2.9cm} | >{\raggedright\arraybackslash}p{2.9cm} | >{\raggedright\arraybackslash}p{2.9cm} | >{\raggedright\arraybackslash}p{2.8cm} |}
   \hline
   \textbf{Use case 2: Foretag måling} &\textbf{Test}& \textbf{Prækondition} & \textbf{Forventet resultat} & \textbf{Godkendt/ kommentar}\\ \hline
   Undtagelse 1: Brugernavn og/eller kode indtastet forkert & Indtast brugernavn "efgh" og kode "1234"& Port valgt. VPN, Personale database og EPJ database er tilsluttet korrekt & Forkert kombinition indtastet &  \\\hline
   &Tryk "Login" & Port valgt. VPN, Personale database og EPJ database er tilsluttet korrekt. & Besked: "Brugernavn og/eller kode indtastet forkert" &\\\hline
\end{tabular}
\end{table}



\begin{table}[H]
\caption{Accepttest for Use case 2}\label{tab:tabel8}
\begin{tabular}{|>{\raggedright\arraybackslash}p{2.5cm}| >{\raggedright\arraybackslash}p{2.9cm} | >{\raggedright\arraybackslash}p{2.9cm} | >{\raggedright\arraybackslash}p{2.9cm} | >{\raggedright\arraybackslash}p{2.8cm} |}
   \hline
   \textbf{Use case 2: Foretag måling} &\textbf{Test}& \textbf{Prækondition} & \textbf{Forventet resultat} & \textbf{Godkendt/ kommentar}\\ \hline
   Udvidelse 1: Slå digitalt filter til/fra:& Tryk på "Digitalt filter OFF" & Signalet er startet & Systemet slår det digitale filter fra. Grafen ses at være ufiltreret(råt) og knappen ændrer navn. &\\\hline
   &Tryk på "Digitalt filter ON" & Signalet er startet & Systemet slår det digitale filter til. Grafen ses at være filtreret og knappen ændrer navn. &\\\hline
\end{tabular}
\end{table}


\begin{table}[H]
\caption{Accepttest for Use case 2}\label{tab:tabel8}
\begin{tabular}{|>{\raggedright\arraybackslash}p{2.5cm}| >{\raggedright\arraybackslash}p{2.9cm} | >{\raggedright\arraybackslash}p{2.9cm} | >{\raggedright\arraybackslash}p{2.9cm} | >{\raggedright\arraybackslash}p{2.8cm} |}
   \hline
   \textbf{Use case 2: Foretag måling } &\textbf{Test}& \textbf{Prækondition} & \textbf{Forventet resultat} & \textbf{Godkendt/ kommentar}\\ \hline
   Udvidelse 2: Juster systolens/diastolens grænseværdi& Tryk på "Systole op"/"Diastole op"& Signalet er startet & Grænseværdien ændres 2.5 mmHg op og intervallet vises på hovedskærmen &\\\hline
   &Tryk på "Systole ned"/"Diastole ned" & Signalet er startet & Grænseværdien ændres 2.5 mmHg ned og intervallet vises på hovedskærmen & \\\hline
\end{tabular}
\end{table}



\begin{table}[H]
\caption{Accepttest for Use case 3}\label{tab:tabel8}
\begin{tabular}{|>{\raggedright\arraybackslash}p{2.5cm}| >{\raggedright\arraybackslash}p{2.9cm} | >{\raggedright\arraybackslash}p{2.9cm} | >{\raggedright\arraybackslash}p{2.9cm} | >{\raggedright\arraybackslash}p{2.8cm} |}
   \hline
   \textbf{Use case 3: Alarmér } &\textbf{Test}& \textbf{Prækondition} & \textbf{Forventet resultat} & \textbf{Godkendt/ kommentar}\\ \hline
   Normalforløb:& Grænseværdi overskrides& Signalet er startet & Alarm starter med lyd og tallet, hvis grænseværdi er overskredet, blinker &\\\hline
\end{tabular}
\end{table}



\begin{table}[H]
\caption{Accepttest for Use case 3}\label{tab:tabel8}
\begin{tabular}{|>{\raggedright\arraybackslash}p{2.5cm}| >{\raggedright\arraybackslash}p{2.9cm} | >{\raggedright\arraybackslash}p{2.9cm} | >{\raggedright\arraybackslash}p{2.9cm} | >{\raggedright\arraybackslash}p{2.8cm} |}
   \hline
   \textbf{Use case 3: Alarmér } &\textbf{Test}& \textbf{Prækondition} & \textbf{Forventet resultat} & \textbf{Godkendt/ kommentar}\\ \hline
   Udvidelse 1: Anden grænseværdi overskrides & Endnu en grænseværdi overskrides & Signalet er er startet og en alarm er startet & Lyd fra første alarm fortsætter og det nye tallet som har overskredet grænseværdien blinker ligeledes &\\\hline
\end{tabular}
\end{table}



\begin{table}[H]
\caption{Accepttest for Use case 3}\label{tab:tabel8}
\begin{tabular}{|>{\raggedright\arraybackslash}p{2.5cm}| >{\raggedright\arraybackslash}p{2.9cm} | >{\raggedright\arraybackslash}p{2.9cm} | >{\raggedright\arraybackslash}p{2.9cm} | >{\raggedright\arraybackslash}p{2.8cm} |}
   \hline
   \textbf{Use case 3: Alarmér } &\textbf{Test}& \textbf{Prækondition} & \textbf{Forventet resultat} & \textbf{Godkendt/ kommentar}\\ \hline
   Udvidelse 2: Udsæt alarm & Tryk på "Udsæt alarm" & Alarmering er startet & Systemet stopper alarmens lyd i et minut &\\\hline
\end{tabular}
\end{table}




\begin{table}[H]
\caption{Accepttest for Use case 4}\label{tab:tabel8}
\begin{tabular}{|>{\raggedright\arraybackslash}p{2.5cm}| >{\raggedright\arraybackslash}p{2.9cm} | >{\raggedright\arraybackslash}p{2.9cm} | >{\raggedright\arraybackslash}p{2.9cm} | >{\raggedright\arraybackslash}p{2.8cm} |}
   \hline
   \textbf{Use case 4: Stop måling } &\textbf{Test}& \textbf{Prækondition} & \textbf{Forventet resultat} & \textbf{Godkendt/ kommentar}\\ \hline
   Normalforløb:& Tryk på "Sluk" & Målingen er foretaget & Målingen, signalet og timer på hovedskærmen stopper &\\\hline
   & Tryk på "Log ud" & Signalet er stoppet & Pop-up vindue kommer op: "Er du sikker?" &\\\hline
   &Tryk "Ja"&Signalet og målingen er stoppet& Startskærmen kommer frem og ny måling kan foretages &\\\hline
\end{tabular}
\end{table}


\begin{table}[H]
\caption{Accepttest for Use case 4}\label{tab:tabel8}
\begin{tabular}{|>{\raggedright\arraybackslash}p{2.5cm}| >{\raggedright\arraybackslash}p{2.9cm} | >{\raggedright\arraybackslash}p{2.9cm} | >{\raggedright\arraybackslash}p{2.9cm} | >{\raggedright\arraybackslash}p{2.8cm} |}
   \hline
   \textbf{Use case 4: Stop måling } &\textbf{Test}& \textbf{Prækondition} & \textbf{Forventet resultat} & \textbf{Godkendt/ kommentar}\\ \hline
Undtagelse 1: Tryk på "Nej" &Tryk "Nej" & Signalet og målingen er stoppet & Kommer tilbage til hovedskærmen &\\\hline
\end{tabular}
\end{table}



\newpage

\newpage

\newpage

\section{Accepttest for ikke-funktionelle krav}

\begin{longtable}{|>{\raggedright\arraybackslash}p{1.1cm}| >{\raggedright\arraybackslash}p{2.7cm} | >{\raggedright\arraybackslash}p{2.7cm} | >{\raggedright\arraybackslash}p{2.7cm} | >{\raggedright\arraybackslash}p{2.2cm} |>{\raggedright\arraybackslash}p{2.2cm}|}
   \caption{Accepttest for ikke-funktionelle krav}\label{tab:label13}
\\ \hline   
\textbf{Krav nr.}&\textbf{Krav} &\textbf{Test}& \textbf{Forventet resultat} & \textbf{Resultat} & \textbf{Godkendt/ kommentar}\\ \hline
  1.1 & Programmet skal have et digitalt filter til udglatning af blodtrykssignal & Send to frekvenser ind (XX Hz og ZZ Hz) & Den ene (XX Hz) af de indsendte frekvenser er blevet fjernet & & \\\hline
  1.2 & Programmet skal give alarm når grænseværdier overskrides med lyd og hvor den overskredede grænse værdi blinker på skærmen. & Overskrid en grænseværdi og tjek alarmering & Alarmen starter& & \\\hline
  1.3 & Programmet skal kunne gemme blodtrykssignalet i en database & Indsend signal og gå ind i databasen og se værdier & Der ligger værdier i databasen & & \\\hline\hline
  2.1 & Programmet skal have to window form: startskærm,der fungerer som EPJ systemet, og hovedskærm, hvilken fungerer som selve blodtryksmåleren & Start program og tjek dette & Der er to window forms & & \\\hline
  2.2 & Programmet skal have en "Login" knap på startskærmen & Start program og tjek startskærm & Startskærmen har en "Login" knap & & \\\hline
  2.3 & Programmet skal have en "Kalibrering" knap på startskærmen & Start program og tjek startskærm & Startskærmen har en "Kalibrering" knap & & \\\hline
  2.4 &Sundhedsfaglig personale skal kunne ændre "device/enhedsnavn" i dropdown på startskærm & Start program og tjek startskærm & Der er en opsætnings dropdown på startskærmen & & \\\hline
  2.5 & Programmet skal indeholde en dropdown, hvor patienten kan vælges på startskærmen & Start program og tjek startskærm & Startskærmen har en dropdown med patienter & & \\\hline
  2.6 & Programmet skal have en "Nulpunkts indstilling" knap på hovedskærmen & Start program og tjek hovedskærm & Der er en "Nulpunkts indstilling" knap på hovedskærmen & & \\\hline
  2.7 & Programmet skal have en knap, til at slå det digitale filter fra og til, på hovedskærmen & Start program og tjek hovedskærm & Der er en "Digital filter" knap på hovedskærmen & & \\\hline
  2.8 & Programmet skal have knapper, til at justere systolisk og diastolisk grænseværdiintervaller op og ned, på hovedskærmen & Start program og tjek hovedskærm & Der er ialt fire knapper, som justerer grænseværdierne på hovedskærmen & & \\\hline
  2.9 & Programmet skal have en "Udsæt alarm" knap på hovedskærmen & Start program og tjek hovedskærm & Der er en "Udsæt alarm" på hovedskærmen & & \\\hline
  2.10 & Programmet skal have en "Tænd" knap på hovedskærmen & Start program og tjek hovedskærm & Hovedskærmen har en "Tænd" kanp& & \\\hline
  2.11 & Programmet skal have en "Sluk" knap på hovedskærmen & Tjek hovedskærm & Hovedskærmen har en "Sluk" knap & & \\\hline
  2.12 & Programmet skal have en "Afbryd" knap på hovedskærmen & Start program og tjek hovedskærm & Der er en "Afbryd" knap på hovedskærmen & & \\\hline
  2.13 & Teksten på startskærmen skal kunne aflæs fra 1 meters afstand med en synsstyrke i intervallet +/-1 & 10 personer med synsstyrke i intervallet +/-1 skal teste startskærmen  & Alle 10 personer kan læse teksten tydeligt & & \\\hline
  2.14 & Teksten og graferne på hovedskærmen skal kunne læses fra 2 meters afstand ved synsstyrke i intervellet på +/-1 & 10 personer med synsstyrke i intervallet +/-1 skal teste hovedskærmen & Alle 10 personer kan læse grafer og teksten på hovedskærmen & & \\\hline
  2.15 & Programmet skal præsentere grafer efter standard & Start program og tjek farver & farverne på grafen er efter standard & & \\\hline
  2.16 & Programmet skal præsentere data i tal efter standard & Start program og tjek at talværdiernes farve er efter standard & Talværdiernes farve er efter standard & & \\\hline\hline
  3.1 & Ingen krav endnu & & & & \\\hline\hline
  4.1 & Tiden der går før målingen af data påbegynder/vises i grafer må maksimalt være 2.0 sek. & Stopur igangsættes samtidig med at signalet tændes & Stopuret viser 2 sek. eller mindre & & \\\hline
  4.2 & Tiden der går fra at data er analyseret til at data er gemt i database må være 2.0 sek. med en tolerance på +/- 15\% & - & & & \\\hline\hline
  5.1 & Softwaren skal være opbygget efter trelagsmodellen & Se programopbygningen & Softwaren er opbygget efter trelagsmodellen & & \\\hline\hline
  6.1 & Der skal være adgang til en computer med Windows 7, 8 eller 10 - computeren skal minimum have 4 GB RAM & & & & \\\hline
  6.2 & Der skal være adgang til en computer hvor National Instruments er installeret & & & & \\\hline
\end{longtable}

\section{Godkendelses formular}
\begin{table}[h!]
\label{tab:tabel14}
\begin{tabular}{| l | >{\raggedright\arraybackslash}p{12cm} |}
   \hline
   \textbf{Dato for test} &\\ \hline
   \textbf{Godkendes af:} & \\ \hline
\end{tabular}
\end{table}
\textbf{Ved underskrivelse af dette dokument godkendes den kørte accepttest}
\newline
\textbf{Sted og dato:}\\
\\
\\
\begin{table}
[h!]
\begin{tabular}{ l lllllllll l}
--------------------------------------&&&&&&&&&&--------------------------------------\\ 
Kundens underskrift &&&&&&&&&&Leverandørens underskrift\\
\end{tabular}
\end{table}

\chapter{Referencer}
\subsubsection{Bøger}
\subsubsection{Internetsider}
\subsubsection{Dokumenter}
