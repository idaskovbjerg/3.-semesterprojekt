\chapter{Kravspecifikation}
\section{Godkendelsesformular}
\begin{table}[h!]
\label{tab:tabel2}
\begin{tabular}{| l | >{\raggedright\arraybackslash}p{12cm} |}
   \hline
   \textbf{Forfattere} & Line, Mette, Brian, Mohamed, Khaled og Ida\\ \hline
   \textbf{Godkendes af:} & Samuel Alberg Thrysøe\\ \hline
   \textbf{Antal sider:} & \\ \hline
   \textbf{Kunde:} & IHA\\ \hline
\end{tabular}
\end{table}
\textbf{Ved underskrivelse af dette dokument accepteres det af begge parter, som værende kravene til udviklingen af det ønskede system.}
\newline
\textbf{Sted og dato:}\\
\\
\\
\begin{table}
[h!]
\begin{tabular}{ l lllllllll l}
--------------------------------------&&&&&&&&&&--------------------------------------\\ 
Kundens underskrift &&&&&&&&&&Leverandørens underskrift\\
\end{tabular}
\end{table}
\section{Use cases}
\begin{table}[h!]
\caption{Use case 1}\label{tab:tabel3}
\begin{tabular}{| l | >{\raggedright\arraybackslash}p{11cm} |}
   \hline
   \textbf{Use case 1} & \textbf{Kalibrer signal}\\ \hline
   Mål: & Få kalibreret signalerne \\ \hline
   Initiering: & Startes af Sundhedsfaglig personale\\ \hline
   Aktører:& Sundhedsfaglig personale (primær), System(sekundær)\\ \hline
   Referencer: & \\ \hline
   Samtidige forekomster: & \\\hline
   Forudsætninger: & \\ \hline
   Resultat:& Signalet er kalibreret\\ \hline
   Hovedscenarie:& 
1. Tryk på "Kalibrering"\newline
2. Systemet starter kalibreringen\newline
3. Besked: "Kalibreringen er fuldendt" vises på GUI\\\hline
Udvidelse/undtagelser: & -\\\hline
\end{tabular}
\end{table}

\begin{table}[h!]
\caption{Use case 2}\label{tab:tabel3}
\begin{tabular}{| l | >{\raggedright\arraybackslash}p{11cm} |}
   \hline
   \textbf{Use case 2} & \textbf{Nulpunkts indstille signal}\\ \hline
   Mål: & Få nulpunkts indstillet signalerne, sådan at signalerne ligger korrekte på deres akse. \\ \hline
   Initiering: & Startes af Sundhedsfaglig personale\\ \hline
   Aktører:& Sundhedsfaglig personale (primær), System (sekundær)\\ \hline
   Referencer: & \\ \hline
   Samtidige forekomster: & \\\hline
   Forudsætninger: & \\ \hline
   Resultat:& Signalet er nulpunkts indstillet\\ \hline
   Hovedscenarie:& 
1. Tryk på "Nulpunks indstilling"\newline
2. Systemet starter nulpunkts indstillingen\newline
3. Besked "Nulpunkts indstillingen er fuldent" vises på GUI\\\hline
Udvidelse/undtagelser: & -\\\hline
\end{tabular}
\end{table}

\begin{table}[h!]
\caption{Use case 3}\label{tab:tabel3}
\begin{tabular}{| l | >{\raggedright\arraybackslash}p{11cm} |}
   \hline
   \textbf{Use case 3} & \textbf{Start måling}\\ \hline
   Mål: & Få indsendt signalerne fra patienten, startet analysen samt skiftet til hovedskærmen \\ \hline
   Initiering: & Startes af Sundhedsfaglig personale\\ \hline
   Aktører:& Sundhedsfaglig personale (primær), Patient (sekundær), System (sekundær)\\ \hline
   Referencer: & \\ \hline
   Samtidige forekomster: & Én patient pr. måling \\\hline
   Forudsætninger: & \\ \hline
   Resultat:& Patiens data vises i GUI\\ \hline
   Hovedscenarie:& 
1. \newline
2. \newline
3. \newline
4. \\\hline
Udvidelse/undtagelser: & -\\\hline
\end{tabular}
\end{table}

\begin{table}[h!]
\caption{Use case 4}\label{tab:tabel3}
\begin{tabular}{| l | >{\raggedright\arraybackslash}p{11cm} |}
   \hline
   \textbf{Use case 4} & \textbf{Gem data}\\ \hline
   Mål: &  Få gemt EKG, blodtrykskurve, puls, systole, diastole og middeltryk i databasen \\ \hline
   Initiering: & Startes af Sundhedsfaglig personale\\ \hline
   Aktører:& Sundhedsfaglig personale (primær), System (sekundær)\\ \hline
   Referencer: & Use Case 3, Use case 11-14\\ \hline
   Samtidige forekomster: & \\\hline
   Forudsætninger: & Use case 3 og Use case 11-14 er kørt succesfuldt  \\ \hline
   Resultat:& Patientes data er gemt i database\\ \hline
   Hovedscenarie:& 
1. Tryk på "Gem"\newline
2. Indtast patientens CPR i pop-up vindue\newline
3. Tryk på "Næste"\newline
4. Systemet gemmer EKG, blodtrykskurve, puls, systole, diastole og middeltryk i database \\\hline
Udvidelse/undtagelser: & -\\\hline
\end{tabular}
\end{table}

\begin{table}[h!]
\caption{Use case 5}\label{tab:tabel3}
\begin{tabular}{| l | >{\raggedright\arraybackslash}p{11cm} |}
   \hline
   \textbf{Use case 5} & \textbf{Udsæt alarm}\\ \hline
   Mål: & Få udsat alarmens lyd i et minut \\ \hline
   Initiering: & Startes af Sundhedsfaglig personale\\ \hline
   Aktører:&Sundhedsfaglig personale (primær), System (sekundær) \\ \hline
   Referencer: & \\ \hline
   Samtidige forekomster: & \\\hline
   Forudsætninger: & Use case 10: Alarm, er igangsat \\ \hline
   Resultat:& Alarmens lyd er stoppet et minut\\ \hline
   Hovedscenarie:& 
1. Tryk på "Udsæt alarm"\newline
2. Systemet stopper alarmens lyd i et minut \\\hline
Udvidelse/undtagelser: & -\\\hline
\end{tabular}
\end{table}

\begin{table}[h!]
\caption{Use case 6}\label{tab:tabel3}
\begin{tabular}{| l | >{\raggedright\arraybackslash}p{11cm} |}
   \hline
   \textbf{Use case 6} & \textbf{Digitalt filter}\\ \hline
   Mål: &  Få slået det digitale filter til og fra \\ \hline
   Initiering: & Startes af Sundhedsfaglig personale\\ \hline
   Aktører:& Sundhedsfaglig personale (primær), System (sekundær)\\ \hline
   Referencer: & \\ \hline
   Samtidige forekomster: & \\\hline
   Forudsætninger: & \\ \hline
   Resultat:& Det digitale filter er slået til eller fra\\ \hline
   Hovedscenarie:& 
1. Tryk på "Digitalt filter OFF" \newline
2. Systemet slår det digitale filter fra\newline
3. Tryk på "Digitalt filter ON"\newline
4. Systemet slår det digitale filter til\\\hline
Udvidelse/undtagelser: & -\\\hline
\end{tabular}
\end{table}

\begin{table}[h!]
\caption{Use case 7}\label{tab:tabel3}
\begin{tabular}{| l | >{\raggedright\arraybackslash}p{11cm} |}
   \hline
   \textbf{Use case 7} & \textbf{Juster systolens grænseværdi}\\ \hline
   Mål: & Få justeret grænseværdierne for systolen op og ned \\ \hline
   Initiering: & Startes af Sundhedsfaglig personale\\ \hline
   Aktører:& Sundhedsfaglig personale (primær), System (sekundær)\\ \hline
   Referencer: & \\ \hline
   Samtidige forekomster: & \\\hline
   Forudsætninger: & \\ \hline
   Resultat:& Grænseværdien for systolen er justeret\\ \hline
   Hovedscenarie:& 
1. Tryk på "Systole op"\newline
2. Grænseværdien ændres 2.5mmHg op og intervallet vises i GUI\newline
3. Tryk på "Systole ned"\newline
4. Grænseværdien ændres 2.5mmHg ned og intervellet vises i GUI\\\hline
Udvidelse/undtagelser: & -\\\hline
\end{tabular}
\end{table}

\begin{table}[h!]
\caption{Use case 8}\label{tab:tabel3}
\begin{tabular}{| l | >{\raggedright\arraybackslash}p{11cm} |}
   \hline
   \textbf{Use case 8} & \textbf{Juster diastolens grænseværdi}\\ \hline
   Mål: &  Få justeret grænseværdierne for diastolen op og ned\\ \hline
   Initiering: &Startes af Sundhedsfaglig personale \\ \hline
   Aktører:& Sundhedsfaglig personale (primær), Systmet (sekundær) \\ \hline
   Referencer: & \\ \hline
   Samtidige forekomster: & \\\hline
   Forudsætninger: & \\ \hline
   Resultat:& Grænseværdien for diastolen er justeret\\ \hline
   Hovedscenarie:& 
1. Tryk "Diastole op"\newline
2. Diastolens grænseværdi ændres 2.5mmHg op og intervellet vises i GUI\newline
3. Tryk "Diastole ned"\newline
4. Diastolens grænseværdi ændres 2.5mmHg ned og intervellet vises i GUI\\\hline
Udvidelse/undtagelser: & -\\\hline
\end{tabular}
\end{table}

\begin{table}[h!]
\caption{Use case 9}\label{tab:tabel3}
\begin{tabular}{| l | >{\raggedright\arraybackslash}p{11cm} |}
   \hline
   \textbf{Use case 9} & \textbf{Stop signalet}\\ \hline
   Mål: & Få stoppet signalet og vendt tilbage til startskærmen. \\ \hline
   Initiering: & Startes af Sundhedsfaglig personale\\ \hline
   Aktører:& Sundhedsfaglig personale (primær), Systemet (sekundær) \\ \hline
   Referencer: & \\ \hline
   Samtidige forekomster: & \\\hline
   Forudsætninger: & \\ \hline
   Resultat:& Stoppet signalet og vendt tilbage til startskærmen \\ \hline
   Hovedscenarie:& 
1. Tryk på "STOP" \newline
2. Systemet stopper signalet og fryser billedet\newline
3. "STOP" ændrer navn til "Nulstil"\newline
4. Tryk på "Nulstil"\newline
5. Startkærmen kommer frem og ny måling kan foretages\\\hline
Udvidelse/undtagelser: & -\\\hline
\end{tabular}
\end{table}


\begin{table}[h!]
\caption{Use case 10}\label{tab:tabel3}
\begin{tabular}{| l | >{\raggedright\arraybackslash}p{11cm} |}
   \hline
   \textbf{Use case 10} & \textbf{Alarmer}\\ \hline
   Mål: & Få startet alarmeringen ved overskridelse af grænseværdier \\ \hline
   Initiering: & Systemet starter denne Use case\\ \hline
   Aktører:& System (primær)\\ \hline
   Referencer: & \\ \hline
   Samtidige forekomster: & \\\hline
   Forudsætninger: & \\ \hline
   Resultat:& Alarmen starter\\ \hline
   Hovedscenarie:& 
1. Systemet tjekker signalet systoliske og diastoliske grænseværdi\newline
2. Grænseværdi overskrides\newline
3. Alarm starter med lyd og lys \\\hline
Udvidelse/undtagelser: & -\\\hline
\end{tabular}
\end{table}


\begin{table}[h!]
\caption{Use case 11}\label{tab:tabel3}
\begin{tabular}{| l | >{\raggedright\arraybackslash}p{11cm} |}
   \hline
   \textbf{Use case 11} & \textbf{Beregn puls}\\ \hline
   Mål: & Få beregnet puls ud fra algoritme \\ \hline
   Initiering: & Systemet starter denne Use case\\ \hline
   Aktører:& System (primær)\\ \hline
   Referencer: & Use case 3 \\ \hline
   Samtidige forekomster: & \\\hline
   Forudsætninger: & Målingen i Use case 3 er kørt succesfuldt\\ \hline
   Resultat:& Patientens puls er beregnet og vises i GUI\\ \hline
   Hovedscenarie:& 
1. Systemet beregner puls\newline
2. Systmet udskriver resultatet i GUI\\\hline
Udvidelse/undtagelser: & -\\\hline
\end{tabular}
\end{table}


\begin{table}[h!]
\caption{Use case 12}\label{tab:tabel3}
\begin{tabular}{| l | >{\raggedright\arraybackslash}p{11cm} |}
   \hline
   \textbf{Use case 12} & \textbf{Beregn systole}\\ \hline
   Mål: & Få beregnet systolens værdi ud fra blodtrykskurven\\ \hline
   Initiering: & Systemet starter denne Use case\\ \hline
   Aktører:& System (primær)\\ \hline
   Referencer: & \\ \hline
   Samtidige forekomster: & \\\hline
   Forudsætninger: & Signalet er indsendt succesfuldt i Use case 3 \\ \hline
   Resultat:& Patientens systole er beregnet og resultatet vises i GUI\\ \hline
   Hovedscenarie:& 
1. Systemet beregner systole \newline
2. Resultatet udskriver resultatet i GUI \\\hline
Udvidelse/undtagelser: & -\\\hline
\end{tabular}
\end{table}


\begin{table}[h!]
\caption{Use case 13}\label{tab:tabel3}
\begin{tabular}{| l | >{\raggedright\arraybackslash}p{11cm} |}
   \hline
   \textbf{Use case 13} & \textbf{Beregn diastole}\\ \hline
   Mål: & Få beregnet diastolens værdi ud fra blodtrykskurven \\ \hline
   Initiering: & Systemet starter denne Use case\\ \hline
   Aktører:& System (primær)\\ \hline
   Referencer: & \\ \hline
   Samtidige forekomster: & \\\hline
   Forudsætninger: & Signalet er indsendt succesfuldt i Use case 3 \\ \hline
   Resultat:& Patientens diastole er beregnet og resultatet vises i GUI\\ \hline
   Hovedscenarie:& 
1. Systmet beregner diastole \newline
2. Systemet udskriver resultatet i GUI \\\hline
Udvidelse/undtagelser: & -\\\hline
\end{tabular}
\end{table}


\begin{table}[h!]
\caption{Use case 14}\label{tab:tabel3}
\begin{tabular}{| l | >{\raggedright\arraybackslash}p{11cm} |}
   \hline
   \textbf{Use case 14} & \textbf{Beregn middeltryk}\\ \hline
   Mål: & Få beregnet middeltryks værdi ud fra blodtrykskurven \\ \hline
   Initiering: & Systemet starter denne Use case\\ \hline
   Aktører:& System (primær)\\ \hline
   Referencer: & \\ \hline
   Samtidige forekomster: & \\\hline
   Forudsætninger: & Signalet er indsendt succesfuldt i Use case 3 \\ \hline
   Resultat:& Patientens middeltryk er beregnet og resultatet vises i GUI\\ \hline
   Hovedscenarie:& 
1. Systemet beregner middeltryk\newline
2. Systemet udskriver resultatet i GUI \\\hline
Udvidelse/undtagelser: & -\\\hline
\end{tabular}
\end{table}


\begin{table}[h!]
\caption{Use case 15}\label{tab:tabel3}
\begin{tabular}{| l | >{\raggedright\arraybackslash}p{11cm} |}
   \hline
   \textbf{Use case 15} & \textbf{Opsætning}\\ \hline
   Mål: &  \\ \hline
   Initiering: & Sundhedsfaglig personale starter denne Use case\\ \hline
   Aktører:& Sundhedsfaglig personale (primær) \\ \hline
   Referencer: & \\ \hline
   Samtidige forekomster: & \\\hline
   Forudsætninger: & \\ \hline
   Resultat:&\\ \hline
   Hovedscenarie:& 
1. \newline
2. \newline
3. \newline
4. \\\hline
Udvidelse/undtagelser: & -\\\hline
\end{tabular}
\end{table}
