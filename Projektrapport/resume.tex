\textbf{Resumé}\\
Formålet med dette projekt er at udvikle et system, som kan tilsluttes et væskefyldt kateter og vise en blodtrykskurve på en computerskærm. I klinisk praksis er der behov for kontinuert at vise patientens blodtryk og herfra aflæse systolen og diastolen. Denne patient overvågning sker primært på intensive afdelinger og operationsstuer. Blodtrykket indikerer patientens helbredstilstand og giver herved det sundhedsfaglige personale en sikkerhed omkring patientens helbred.\\\\ Dokumentationen for projektet er struktureret efter ASE-modellen, systemet er udviklet ud fra V-modellen og software samt hardware processen er udarbejdet gennem, en iterativ agil udvikling med vandfaldsmodellen som udgangspunkt. I dette projekt er der valgt en løsning, hvor blodtrykket præsenteres grafisk og værdien for systole, diastole samt middeltryk vises. Der er også udarbejdet en database, hvor både patients data og personalets data ligger gemt. I denne database gemmes patientens blodtryk løbende.
\\\\
Der er blevet udviklet et system, som kan præsenterer et blodtrykssignal og alarmere hvis dette overskrider de angivende grænseværdier. Blodtrykket er varierende hos forskellige patienter grupper, og disse grænseværdier vil derfor kunne reguleres op og ned. Det kan i dette projekt konkluderes, at vi har udarbejdet et hardware og et software produkt, som ved interaktion kan indhente data og præsentere disse på en brugergrænseflade. Det er lykkedes os at få et blodtrykssignal vist, dog har det tidsmæssigt ikke været muligt, at implementerer EKG og ressourcemæssigt ikke været muligt at implementerer iltmætning. \\\\
\textbf{Abstract}\\
The purpose of this project is to develop a system that can be connected to a liquid-filled catheter and display a blood pressure curve on a computer screen. In clinical practice, it is necessary to continuously display the patient blood pressure and hereby read the systolic and diastolic value. This monitoring of patient occurs in intensive care and operating rooms. Blood pressure is a indicator of the patient’s state of health and gives the health professionals evidence about the patient’s state of health. \\
The ASE model structures the documentation for the project and the system has been developed by the V model. The software and hardware process is developed through an iterative agile process where the waterfall model has been a starting point.\\\\
In this project we have chosen a solution where blood pressure is presented graphically and where the value for systolic, diastolic and mean pressure are shown. Additionally a database has been made where both patient data and healthcare professional data are stored. This database stores patient blood pressure regularly. The system has been developed, which can provide a blood pressure signal and an alarm starts if the blood pressure exceeds the limit for normal blood pressure. Blood pressure varies in different patient groups and therefor it is possible to change the limit. \\\\
For the project we have succeeded in making a product where in interaction between hardware and software can obtain present those on a computer screen. 
It has been possible for us to show a blood pressure signal, however from a time consuming point of view it has not been possible to implement EKG and due to lake of sufficient resources we have not been able to implement oxygen saturation.  